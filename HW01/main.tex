\documentclass{article}
\usepackage{amsmath,amsthm,latexsym,paralist,url}
\usepackage[margin=1in]{geometry}


\theoremstyle{definition}
\newtheorem{problem}{Problem}
\newtheorem*{solution}{Solution}
\newtheorem*{resources}{Resources}


\newcommand{\honor}{\noindent \textbf{Aggie Honor Statement: }On my honor as an Aggie, I have neither
  given nor received any unauthorized aid on any portion of the academic work included in this assignment.
}

 
\newcommand{\checklist}{\noindent\textbf{Checklist:}
\begin{compactenum}
\item Did you abide by the Aggie Honor Code?
\item Did you solve all problems and start a new page for each? 
\item Did you submit the PDF to eCampus?
\end{compactenum}
}

\newcommand{\problemset}[1]{\begin{center}\textbf{Problem Set #1}\end{center}}
\newcommand{\duedate}[1]{\begin{quote}\textbf{Due: #1} on eCampus (\url{ecampus.tamu.edu}).\end{quote}}
\newcommand{\mysectionnumber}[0]{503}

\title{CSCE 222: Discrete Structures for Computing\\Section 503\\Fall 2016}
\author{Joseph Martinsen}

\begin{document}

\maketitle

\problemset{1}

\duedate{4 September 2016 (Sunday) before 11:59 p.m.}

\bigskip

% Algorithms
\begin{problem} (20 points)\\
You have $n$ coins, exactly one of which is counterfeit.  You know counterfeit coins weigh more than authentic coins.  Devise an algorithm for finding the counterfeit coin using a balance scale\footnote{A balance scale compares the weight of objects placed on it.  The result of the comparison is either left side heavier, right side heavier, or both sides equal.}.  Express your algorithm in pseudocode.  For $n=12$, how many weighings does your algorithm use? 
\end{problem}

\begin{solution}

\item \bf procedure $counterfeit\_coin(a_{1}+a_{2}+a_{3}+...+a_{n}$: coin weights$)$

$n=12$

for $i=1\ $to $n$

\qquad $a_{i-1}\rightarrow left\_scale$

\qquad $a_{i}\rightarrow right\_scale$ 

\qquad if $right\_scale<left\_scale$

\qquad \qquad $counterfeit\_location=i$

\qquad else if $left\_scale>right\_scale$

\qquad \qquad $counterfeit\_location=i-1$

return $counterfeit\_location$

\end{solution}

\newpage

% Algorithms
\begin{problem} (20 points)\\
Devise an algorithm that takes as input a list of $n$ integers in unsorted order, where the integers are not necessarily distinct, and outputs the location (index of first element) and length of the longest contiguous non-decreasing subsequence in the list.  If there is a tie, it outputs the location of the first such subsequence.  Express your algorithm in pseudocode.  For the list $9,7,9,4,5,8,1,0,5,9$, what is the algorithm's output? How many comparison operations between elements of the list are used?
\end{problem}

\begin{solution}
\item
\textbf{procedure} \textit{non-decreasing-location}(\textit{$a_1, a_2, a_3,...,a_n$}: integers)\newline
$location:=1$\newline
\textbf{for} $i=2$ \textbf{to} $n-1$\par
	\textbf{if} $a_{i-1} < a_i$ \textbf{then}\par
    	$location:=i$\newline
        \textbf{else if} $a_{i-1} = a_i$ \textbf{then}\newline
        \textbf{return} \textit{non-decreasing-location}\{\textit{non-decreasing-location} is the location of the first element in the subsequence of non-decreacing elements\}

\end{solution}

\newpage

% Growth of Functions - Big O
\begin{problem} (20 points)\\
Arrange the following functions in order such that each function is big-$O$ of the next function:  $2\cdot3^n$, $3n!$, $2019\log{n}$, $\displaystyle \frac{n^3}{10^6}$, $n\log{n}$,  $\sqrt{n}$, $3\cdot2^n$.  Prove your answer is correct by giving the witnesses for each pair of consecutive functions.
\end{problem}

\begin{solution}
The following functions are arranged such that that each function is a big-$%
O $ of the next function.

\begin{enumerate}
\item $2019\log {n}$

$\qquad 2019\log {n\leq C}\sqrt{n}$

$\qquad 2019\log {n\leq 2019}\sqrt{n}\qquad C=2019$

$\qquad \log {n\leq }\sqrt{n}$

$\qquad 0.0\leq 1\qquad \qquad \qquad \qquad k=1$

$\qquad 2019\log {n}$ is $O(\sqrt{n})$ by taking $C=2019$ and $k=1$ as
witnesses. \bigskip

\item $\sqrt{n}$

\qquad $\sqrt{n}\leq C(n\log {n)}$

\qquad $\frac{1}{\sqrt{n}}\leq \log {n\qquad \qquad C=1}$

\qquad $\frac{1}{2}\leq \log 4\qquad \qquad k=4$

\qquad $0.5\leq 1.\,\allowbreak 386\,3$

\qquad $\sqrt{n}$ is $O(n\log {n})$ by taking $C=1$ and $k=4$ as witnesses.
\bigskip

\item $n\log {n}$

$\qquad n\log {n\leq C(}\dfrac{n^{3}}{10^{6}})$

$\qquad n\log {n\leq }\dfrac{n^{3}}{10^{6}}\qquad C=1$

$\qquad 6044\leq 6332\qquad \qquad \qquad \qquad k=1850$

$\qquad n\log {n}$ is $O(\dfrac{n^{3}}{10^{6}})$ by taking $C=1$ and $k=1850$
as witnesses. \bigskip 

\item $\dfrac{n^{3}}{10^{6}}$

$\qquad \dfrac{n^{3}}{10^{6}}\leq C(3\cdot 2^{n})$

$\qquad \dfrac{n^{3}}{10^{6}}\leq 3\cdot 2^{n}\qquad C=1$

$\qquad 0\leq 3\qquad \qquad \qquad k=0$

$\qquad \dfrac{n^{3}}{10^{6}}$ is $O(3\cdot 2^{n})$ by taking $C=1$ and $k=0$
as witnesses. \bigskip 

\item $3\cdot 2^{n}$

\qquad $3\cdot 2^{n}\leq C(2\cdot 3^{n})$

\qquad $3\cdot 2^{n}\leq \dfrac{3}{2}(2\cdot 3^{n})\qquad C=\dfrac{3}{2}$

\qquad $3\cdot 2^{n}\leq 3\cdot 3^{n}$

\qquad $2^{n}\leq 3^{n}$

\qquad $2\leq 3\qquad \qquad \qquad k>1$

\qquad $3\cdot 2^{n}$ is $O(2\cdot 3^{n})$ by taking $C=\dfrac{3}{2}$ and $%
k>1$ as witnesses.

\item $2\cdot 3^{n}$

\qquad $2\cdot 3^{n}\leq C(3n!)$

\qquad $2\cdot 3^{n}\leq \dfrac{2}{3}(3n!)\qquad C=\dfrac{2}{3}$

\qquad $2\cdot 3^{n}\leq 2\cdot n!$

\qquad $3^{n}\leq n!$

\qquad $2187\leq 5040\qquad \qquad \qquad k>7$

$\qquad 2\cdot 3^{n}$ is $O(3n!)$ by taking $C=\dfrac{2}{3}$ and $k>7$ as
witnesses$\qquad \qquad $ \bigskip \newline

\item $3n!$ \bigskip 

\qquad $3n!$ is $O(n!)$ with $C=4$ and $k>0$ as witnesses
\end{enumerate}
\end{solution}

\newpage

% Growth of Functions - Big O
\begin{problem} (20 points)\\
For each of the following functions, give a big-$O$ estimate, including witnesses, using a simple function $g(n)$ of the smallest order:
\begin{enumerate}
\item $(n^2+8)(n+1)$
\item $(n\log{n} + 1)^2+(\log{n}+1)(n^2+1)$
\item $\displaystyle n^{2^n}+n^{n^2}$
\item $\displaystyle \frac{n^4+5\log{n}}{x^3+1}$
\item $2x^4+7x^3+5x+3$
\end{enumerate}
\end{problem}

\begin{solution}
\item \begin{enumerate}
\item $(n^{2}+8)(n+1)$

\qquad $(n^{2}+8)(n+1)\leq C(n^{3})$

$\qquad n^{3}+n^{2}+8n+8\leq C(n^{3}+n^{3}+n^{3}+n^{3})$

\qquad $n^{3}+n^{2}+8n+8\leq 4(n^{3})\qquad C=4$

\qquad $68\leq 108\qquad \qquad \qquad \qquad \qquad k>3$

\qquad Thus $n^{3}$ is $O((n^{2}+8)(n+1))$ by taking $C=4$ and $k>3$ as
witnesses.

\item $(n\log {n}+1)^{2}+(\log {n}+1)(n^{2}+1)$

\qquad $(n\log {n}+1)^{2}+(\log {n}+1)(n^{2}+1)\leq C(n^{3})$

\qquad $(n\log {n}+1)(\log {n}+1+n^{2}+1)\leq C(n^{3})$

\qquad $(n\log {n})^{2}+n\log {n+}n\log {n+1+n}^{3}\log {n}+n^{2}+n\log
n+1\leq C()$

\qquad $n^{3}\log n+(n\log {n})^{2}+5n\log n+n^{3}+2n^{2}+2\leq
C(n^{3}+n^{3}+n^{3}+n^{3}+n^{3})$

\qquad $n^{3}\log n+(n\log {n})^{2}+5n\log n+n^{3}+2n^{2}+2\leq 5n^{3}\qquad
C=5$

\qquad $8.06\leq 40\qquad k>3$

\qquad Thus $(n\log {n}+1)^{2}+(\log {n}+1)(n^{2}+1)$ is $O(n^{3})$ with $C=5
$ and $k>3$ as witnesses.

\qquad 

\item $n^{2^{n}}+n^{n^{2}}$

\qquad $n^{2^{n}}+n^{n^{2}}\leq C(n^{2^{n}})$

\qquad $n^{2^{n}}+n^{n^{2}}\leq 2n^{2^{n}}\qquad \qquad \qquad C=2$

\qquad $2.33\cdot 10^{22}\leq 4.66\cdot 10^{22}\qquad k>5$

\qquad Thus $n^{2^{n}}+n^{n^{2}}$ is $O(n^{2^{n}})$ with $C=2$ and $k>5$ as
witnesses.

\item $\dfrac{n^{4}+5\log {n}}{n^{3}+1}$

\qquad $\dfrac{n^{4}+5\log {n}}{n^{3}+1}\leq C(n)$

\qquad $\dfrac{n^{4}}{n^{3}+1}+\dfrac{5\log {n}}{n^{3}+1}\leq 2n\qquad C=2$

\qquad $5\leq 10\qquad \qquad k>5$

\qquad Thus $\dfrac{n^{4}+5\log {n}}{n^{3}+1}$ is $O(x)$ with $C=2$ and $k>5$
as witnesses

\item $2x^{4}+7x^{3}+5x+3$

\qquad $2x^{4}+7x^{3}+5x+3\leq 2x^{4}+x^{4}+x^{4}+x^{4}$

$\qquad 2x^{4}+7x^{3}+5x+3\leq 5x^{4}\qquad \qquad C=5$

$\qquad 369\leq 405\qquad \qquad k>3$\newline

\qquad Thus $2x^{4}+7x^{3}+5x+3$ is $O(x^{4})$ with $C=5$ and $k>3$ as
witnesses
\end{enumerate}


\end{solution}

\newpage

% Growth of Functions - Big Theta
\begin{problem} (20 points)\\
For each of the following functions, determine whether that function is of the same order as $n^2$ either by finding witnesses or showing that sufficient witnesses do not exist:
\begin{enumerate}
\item $13n+12$
\item $n^2+1000 n\log{n}$
\item $3^n$
\item $3n^2+n-5$
\item $\displaystyle \frac{n^3+2n^2-n+3}{4n}$
\end{enumerate}
\end{problem}



\begin{solution}

\item Suppose that there are constants $C$ and $k$ for which $n^2 \leq C(13n+12)$ whenever $n > k$. We can divide both 
		sides of $Cn^2 \leq 13n+12$ by $n$ and simplify to $n \leq 3C$ for values of $n > k$. However, no matter what 
        $C$ and $k$ are, the inequality $n \leq 3C$ cannot hold for all $n$ with $n > k$. In particular, once we set a 
        value of $k$, we see that when $n$ is larger than the maximum of $k$ and $C$, it is not true that $n \leq C$ 
        even though $n > k$. This contradiction shows that $n^2$ is not $O(13n+12)$. Thus the function $13n+12$ is not 
        as the same order as $n^2$.
\item Suppose that there are particular witnesses $C$ and $k$ for which $n^2+1000n\log{n} \leq Cn^2$. After dividing both sides by $n$ the resultant is $ n + \log{{n}^{1000}}\leq Cn$.
\item $3^n$
\item $3n^2+n-5$
\item $\displaystyle \frac{n^3+2n^2-n+3}{4n}$

\end{solution}

\newpage


\bigskip
\honor

\bigskip
\checklist
\end{document}
