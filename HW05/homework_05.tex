\documentclass{article}
\usepackage{amsmath,amssymb,amsthm,latexsym,paralist,url}
\usepackage[margin=1in]{geometry}

\theoremstyle{definition}
\newtheorem{problem}{Problem}
\newtheorem*{solution}{Solution}
\newtheorem*{resources}{Resources}


\newcommand{\honor}{\noindent \textbf{Aggie Honor Statement: }On my honor as an Aggie, I have neither
  given nor received any unauthorized aid on any portion of the academic work included in this assignment.
}


\newcommand{\checklist}{\noindent\textbf{Checklist:}
Did you...
\begin{compactenum}
\item abide by the Aggie Honor Code?
\item solve all problems?
\item start a new page for each problem?
\item show your work clearly?
\item type your solution?
\item submit a PDF to eCampus?
\end{compactenum}
}

\newcommand{\problemset}[1]{\begin{center}\textbf{Problem Set #1}\end{center}}
\newcommand{\duedate}[1]{\begin{quote}\textbf{Due: #1} on eCampus (\url{ecampus.tamu.edu}). \\You must show your work in order to recieve credit.\end{quote}}
\newcommand{\mysectionnumber}[0]{503}

\title{CSCE 222: Discrete Structures for Computing\\Section \mysectionnumber\\Fall 2016}
\author{Joseph Martinsen}

\begin{document}

\maketitle

\problemset{5}

\duedate{2 October 2016 (Sunday) before 11:59 p.m.}

\bigskip

% Sets and Set Operations
\begin{problem} (25 points)\\
Suppose that $A$, $B$, and $C$ are sets. Prove or disprove that $(A-B)-C=(A-C)-B$.
\end{problem}

\begin{solution}\ \\


\end{solution}

\newpage

% Sets and Set Operations
\begin{problem} (25 points)\\
Determine whether the symmetric difference is associative; that is, if $A$, $B$, and $C$ are sets, does it follow that $A \oplus (B \oplus C) = (A \oplus B) \oplus C$?
\begin{enumerate}
\item[a.] Use a Venn diagram.
\item[b.] Use a membership table.
\item[c.] Use set identities.
\end{enumerate}
\end{problem}

\begin{solution}\ \\



\end{solution}

\newpage

% functions
\begin{problem} (25 points)\\
Determine whether $f$ is a function from $\mathbb{Z}$ to $\mathbb{R}$ if
\begin{enumerate}
\item[a.] $f(n) = \pm n$
\item[b.] $\displaystyle f(n) = \left\lceil{n \over 2}\right\rceil$
\item[c.] $f(n) = \sqrt{n^2+1}$
\item[d.] $f(n) = \sqrt{n}$
\item[e.] $\displaystyle f(n) = {1 \over n^2 - 4}$
\end{enumerate}
\end{problem}

\begin{solution}\ \\



\end{solution}

\newpage

% functions
\begin{problem} (25 points)\\
Consider the function $f: \mathbb{Z} \to (\mathbb{N}-\{0\})~\text{ where }~f(n) = \left\{
\begin{array}{ll}
1-2n & n \leq 0\\
2n & n > 0\\
\end{array}\right.$
\begin{enumerate}
\item[a.] Prove that $f$ is a bijection by showing that it is both injective and surjective.
\item[b.] Find the inverse function $f^{-1}$.
\end{enumerate}
\end{problem}

\begin{solution}\ \\




\end{solution}

\newpage

\bigskip
\honor

\bigskip
\checklist
\end{document}
