\documentclass{article}
\usepackage{amsmath,amssymb,amsthm,latexsym,paralist,url}
\usepackage[margin=1in]{geometry}
\usepackage{tikz}
\usetikzlibrary{arrows,automata}

\theoremstyle{definition}
\newtheorem{problem}{Problem}
\newtheorem*{solution}{Solution}
\newtheorem*{resources}{Resources}


\newcommand{\honor}{\noindent \textbf{Aggie Honor Statement: }On my honor as an Aggie, I have neither
  given nor received any unauthorized aid on any portion of the academic work included in this assignment.
}

 
\newcommand{\checklist}{\noindent\textbf{Checklist:}
Did you...
\begin{compactenum}
\item abide by the Aggie Honor Code?
\item solve all problems?
\item start a new page for each problem?
\item show your work clearly?
\item type your solution?
\item submit a PDF to eCampus?
\end{compactenum}
}

\newcommand{\problemset}[1]{\begin{center}\textbf{Problem Set #1}\end{center}}
\newcommand{\duedate}[1]{\begin{quote}\textbf{Due: #1} on eCampus (\url{ecampus.tamu.edu}). \\You must show your work in order to recieve credit.\end{quote}}
\newcommand{\mysectionnumber}[0]{503}

\title{CSCE 222: Discrete Structures for Computing\\Section \mysectionnumber\\Fall 2016}
\author{Joseph Martinsen}

\begin{document}

\maketitle

\problemset{10}

\duedate{6 November 2016 (Sunday) before 11:59 p.m.}

\bigskip

% Mathematical Induction
\begin{problem} (25 points)\\
Use  induction on $n$ to prove that $\displaystyle \sum_{i=0}^{n-1} \frac{i}{2^i} = 2-\frac{n+1}{2^{n-1}}$
\end{problem}

\begin{solution}\ \\

  \begin{align*}
    \displaystyle \sum_{i=0}^{n-1} \frac{i}{2^i} = 2-\frac{n+1}{2^{n-1}}  &\text{ is true for all } n \geq 1 \\
    \text{\underline{Basis Step: }Show } P(1)&\\
    P(1) &= 2-\frac{1+1}{2^{1-1}} \\
    &= 2-\frac{2}{1} = 0\\
    P(1) &= \displaystyle \sum_{i=0}^{1-1} \frac{i}{2^i} \\
    &= \frac{0}{2^0} = 0\\
    \therefore P(0) \text{ holds} \\
    \text{\underline{Inductive Step: }Show }P(k) \rightarrow P(k+1)\\
    \text{Assume } P(k) \text{for arbitrary } k > 1: &\displaystyle \sum_{i=0}^{k-1} \frac{i}{2^i} = 2-\frac{k+1}{2^{k-1}} \\
    \text{Show }P(k+1):& \displaystyle \sum_{i=0}^{k} \frac{i}{2^i} = 2-\frac{k+2}{2^{k}} \\
    \displaystyle \sum_{i=0}^{k} \frac{i}{2^i} &= \frac{k}{2^k} + \displaystyle \sum_{i=0}^{k-1} \frac{i}{2^i} \\
    &= \frac{k}{2^k} + 2-\frac{k+1}{2^{k-1}}&\text{By HI} \\
    &= 2 - \frac{k+2}{2^k} \\
    \therefore P(k) \rightarrow P(k+1) \text{ holds} \\
    \therefore \displaystyle \sum_{i=0}^{n-1} \frac{i}{2^i} = 2-\frac{n+1}{2^{n-1}}  &\text{ holds for all } n \geq 1 \text{ by mathematical induction}
  \end{align*}

\end{solution}

\newpage

% Mathematical Induction
\begin{problem} (25 points)\\
A guest at a party is a \textbf{celebrity} if this person is known by every other guest, but knows none of them. 
There is at most one celebrity at a party\footnote{If there were two, they
would know each other. A particular party may have no celebrity}. 
Your task is to find the celebrity, if one exists, at a party by asking only one type of question --
asking a guest whether they know a second guest.
Everyone must answer your questions truthfully.
That is, if Alice and Bob are two people at the party, you can ask Alice whether she knows Bob; 
she must answer correctly.
Use mathematical induction to show that if there are $n$ people at the party, then you can find the celebrity, if there is one, with $3(n-1)$ questions. 
\textit{Hint: First, ask a question to eliminate one person as a celebrity. 
Then use the inductive hypothesis to identify a potential celebrity.
Finally, ask two more questions to determine whether that person is actually a celebrity.} 
\end{problem}

\begin{solution}\ \\
\underline{Base Step:} if two people $A$ and $B$ are at a paty, you ask if they know one another. If one of the two peole says Yes, and the other one says No, then the person who said No is a celebrity, else there is not a celebrity present \\
\underline{Inductive Step: } Assume that the above statement is true for a party of k people and prove that it is also true for a party of $k+1$ people.\\
Let $A$ and $B$ be party members. Ask $A$ if they know $B$. If they answer Yes, A is not a celebrity. Else, if they answer no then it follows that $B$ is not a celebrity. One party member has now been ruled out as being a celebrity. Going on to the next $n$ party members, use the inductive hypothesis to find the celebrity with $3(n-1)$ questions.\\
If there is a celebrity $C$, ask $C$ if they know the last person. Also ask the last person if they know $C$. If $C$ does not know the last party member and the last party member knows $C$, then $C$ is a celebrity.
\end{solution}

\newpage

% Strong Induction
\begin{problem} (25 points)\\
Determine which Fibonacci numbers are divisible by 3. Use strong induction on $n$ to prove your conjecture.\\
The Fibonacci sequence satisfies the recurrence relation $f_n = f_{n-1} + f_{n-2}$ where $f_0 = 0$ and $f_1 = 1$.
\end{problem}

%\begin{solution}\ \\
%\end{solution}

%\newpage

% Strong Induction
\begin{problem} (25 points)\\
Restaurant 222 offers gift certificates in denominations of \$8 and \$15.  Determine the possible total amounts you can form using these denominations of gift certificates.  Prove your answer using strong induction.
\end{problem}

\begin{solution}\ \\
  \begin{align*}
    P(n) = 
  \end{align*}
\end{solution}


\bigskip
\honor

\bigskip
\checklist
\end{document}
