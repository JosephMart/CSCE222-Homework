\documentclass{article}
\usepackage{amsmath,amssymb,amsthm,latexsym,paralist,url}
\usepackage[margin=1in]{geometry}
\usepackage{tikz}
\usetikzlibrary{arrows,automata}

\theoremstyle{definition}
\newtheorem{problem}{Problem}
\newtheorem*{solution}{Solution}
\newtheorem*{resources}{Resources}


\newcommand{\honor}{\noindent \textbf{Aggie Honor Statement: }On my honor as an Aggie, I have neither
  given nor received any unauthorized aid on any portion of the academic work included in this assignment.
}

 
\newcommand{\checklist}{\noindent\textbf{Checklist:}
Did you...
\begin{compactenum}
\item abide by the Aggie Honor Code?
\item solve all problems?
\item start a new page for each problem?
\item show your work clearly?
\item type your solution?
\item submit a PDF to eCampus?
\end{compactenum}
}

\newcommand{\problemset}[1]{\begin{center}\textbf{Problem Set #1}\end{center}}
\newcommand{\duedate}[1]{\begin{quote}\textbf{Due: #1} on eCampus (\url{ecampus.tamu.edu}). \\You must show your work in order to recieve credit.\end{quote}}
\newcommand{\mysectionnumber}[0]{503}

\title{CSCE 222: Discrete Structures for Computing\\Section \mysectionnumber\\Fall 2016}
\author{Joseph Martinsen}

\begin{document}

\maketitle

\problemset{10}

\duedate{6 November 2016 (Sunday) before 11:59 p.m.}

\bigskip

% Mathematical Induction
\begin{problem} (25 points)\\
Use  induction on $n$ to prove that $\displaystyle \sum_{i=0}^{n-1} \frac{i}{2^i} = 2-\frac{n+1}{2^{n-1}}$
\end{problem}

%\begin{solution}\ \\
%\end{solution}

%\newpage

% Mathematical Induction
\begin{problem} (25 points)\\
A guest at a party is a \textbf{celebrity} if this person is known by every other guest, but knows none of them. 
There is at most one celebrity at a party\footnote{If there were two, they
would know each other. A particular party may have no celebrity}. 
Your task is to find the celebrity, if one exists, at a party by asking only one type of question --
asking a guest whether they know a second guest.
Everyone must answer your questions truthfully.
That is, if Alice and Bob are two people at the party, you can ask Alice whether she knows Bob; 
she must answer correctly.
Use mathematical induction to show that if there are $n$ people at the party, then you can find the celebrity, if there is one, with $3(n-1)$ questions. 
\textit{Hint: First, ask a question to eliminate one person as a celebrity. 
Then use the inductive hypothesis to identify a potential celebrity.
Finally, ask two more questions to determine whether that person is actually a celebrity.} 
\end{problem}

%\begin{solution}\ \\
%\end{solution}

%\newpage

% Strong Induction
\begin{problem} (25 points)\\
Determine which Fibonacci numbers are divisible by 3. Use strong induction on $n$ to prove your conjecture.\\
The Fibonacci sequence satisfies the recurrence relation $f_n = f_{n-1} + f_{n-2}$ where $f_0 = 0$ and $f_1 = 1$.
\end{problem}

%\begin{solution}\ \\
%\end{solution}

%\newpage

% Strong Induction
\begin{problem} (25 points)\\
Restaurant 222 offers gift certificates in denominations of \$8 and \$15.  Determine the possible total amounts you can form using these denominations of gift certificates.  Prove your answer using strong induction.
\end{problem}

%\begin{solution}\ \\
%\end{solution}


\bigskip
\honor

\bigskip
\checklist
\end{document}
