\documentclass{article}
\usepackage{amsmath,amssymb,amsthm,latexsym,paralist,url}
\usepackage[margin=1in]{geometry}
\usepackage{tikz}
\usetikzlibrary{arrows,automata}
% settings for algorithm2e psuedocode style
\usepackage[procnumbered, ruled, linesnumbered, commentsnumbered, noend]{algorithm2e}

\theoremstyle{definition}
\newtheorem{problem}{Problem}
\newtheorem*{solution}{Solution}
\newtheorem*{resources}{Resources}


\newcommand{\honor}{\noindent \textbf{Aggie Honor Statement: }On my honor as an Aggie, I have neither
  given nor received any unauthorized aid on any portion of the academic work included in this assignment.
}

 
\newcommand{\checklist}{\noindent\textbf{Checklist:}
Did you...
\begin{compactenum}
\item abide by the Aggie Honor Code?
\item solve all problems?
\item start a new page for each problem?
\item show your work clearly?
\item type your solution?
\item submit a PDF to eCampus?
\end{compactenum}
}

\newcommand{\problemset}[1]{\begin{center}\textbf{Problem Set #1}\end{center}}
\newcommand{\duedate}[1]{\begin{quote}\textbf{Due: #1} on eCampus (\url{ecampus.tamu.edu}). \\You must show your work in order to recieve credit.\end{quote}}
\newcommand{\mysectionnumber}[0]{503}

\title{CSCE 222: Discrete Structures for Computing\\Section \mysectionnumber\\Fall 2016}
\author{Joseph Martinsen}

\begin{document}

\maketitle

\problemset{12}

\duedate{20 November 2016 (Sunday) before 11:59 p.m.}

\bigskip

% program correctness
\begin{problem} (20 points)\\
Verify that the following program segment is correct with respect to the initial assertion $y=3$ and the final assertion $z=6$.
\begin{algorithm}
\DontPrintSemicolon
\caption{program segment}
$x := 2$\;
$z := x + y$\;
\If{$y>0$}{
	$z:=z+1$\;
}
\Else{
	$z := 0$\;
}
\end{algorithm}
\end{problem}

\begin{solution}\ \\
  \begin{align*}
    p \Rightarrow y &:= 3 \\
    x &:= 2 \\
    z &:= x + y \Rightarrow 5 \\
    \text{Condition: } y &> 0 \\
    z &:= 5 + 1 = 6 \\
    \therefore p\{S\}q \text{ because } z&=6 \wedge z =6 \text{ with given p}
  \end{align*}

\end{solution}

\newpage

% program correctness
\begin{problem} (30 points)\\
Use a loop invariant to prove that the following program segment for computing the $n$-th power, where $n$ is a positive integer, of a real number $x$ is correct.
\begin{algorithm}
\caption{program segment}
\DontPrintSemicolon
$power := 1$\;
$i := 1$\;
\While{$i \leq n$}{
	$power := power * x$\;
	$i := i + 1$\;
}
\end{algorithm}
\end{problem}

\begin{solution}\ \\
To show p is a loop invariant, if p is true at the beginning of the loop, then p must still hold true after the exectutoin of the loop
\begin{center}
\begin{align*}
  \text{Let } p \Rightarrow i \leq n+1 &\wedge power = x^{i-1} \\
  \text{Assume } p \text{ is true,} \\
  \text{Check if $p$ is true at the end of the loop}\\
  i_{loop} &= i+1 \\
  power_{loop} &= power * x\\
   &= x^{i-1} * x \\
   &= x^i \\
   &= x^{i_{loop}-1} \\
   \text{With condition } i &\leq n \\
   i_{loop} &\leq n+1 \text{ follows} \\
   \therefore p &\text{ is true at the end of the loop}\\
   \text{Show $p$ is true before loop is executed}\\
   \text{It is given that } n \geq 1 & \text{ so, } i \leq n+1 \\
   power &= 1 \\
   &= x^0 \\
   &= x^{i-1} \\
   \therefore p \text{ is true before the loop executes}\\
   \text{Show loop terminates correctly}\\
   \text{Loop terminates when $p$ is true and $i \leq n$ is false}\\
   i &= n + 1 \text{ so }\\
   power &= x^{(n+1)-i} \\
   &= x^n \\
   \therefore \text{Loop terminates correctly} \\
   \text{Show loop terminates} \\
   i_0 = 1 \\
   i = i + 1\\
   i \text{ will continue to increace until } i > n \text{ resulting in }i \leq n \text{ being false} \\
   \therefore \text{loop terminates after $n$ iterations} \\
   \therefore \text{program is correct}
\end{align*}
\end{center}
\end{solution}

\newpage


\bigskip
\honor

\bigskip
\checklist
\end{document}
