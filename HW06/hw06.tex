\documentclass{article}
\usepackage{amsmath,amssymb,amsthm,latexsym,paralist,url}
\usepackage[margin=1in]{geometry}
\usepackage{cancel}

\theoremstyle{definition}
\newtheorem{problem}{Problem}
\newtheorem*{solution}{Solution}
\newtheorem*{resources}{Resources}


\newcommand{\honor}{\noindent \textbf{Aggie Honor Statement: }On my honor as an Aggie, I have neither
  given nor received any unauthorized aid on any portion of the academic work included in this assignment.
}


\newcommand{\checklist}{\noindent\textbf{Checklist:}
Did you...
\begin{compactenum}
\item abide by the Aggie Honor Code?
\item solve all problems?
\item start a new page for each problem?
\item show your work clearly?
\item type your solution?
\item submit a PDF to eCampus?
\end{compactenum}
}

\newcommand{\problemset}[1]{\begin{center}\textbf{Problem Set #1}\end{center}}
\newcommand{\duedate}[1]{\begin{quote}\textbf{Due: #1} on eCampus (\url{ecampus.tamu.edu}). \\You must show your work in order to recieve credit.\end{quote}}
\newcommand{\mysectionnumber}[0]{503}

\title{CSCE 222: Discrete Structures for Computing\\Section \mysectionnumber\\Fall 2016}
\author{Joseph Martinsen}

\begin{document}

\maketitle

\problemset{6}

\duedate{9 October 2016 (Sunday) before 11:59 p.m.}

\bigskip

% Sequences and Sums
\begin{problem} (25 points)
\begin{enumerate}
\item Show that $ \sum_{j=1}^n (a_j-a_{j-1})=a_n-a_0$, where $a_0,a_1,\ldots,a_n$ is a sequence of real numbers.\\ \textit{This type of sum is called \textbf{telescoping}.}
\item Sum both sides of the identity $k^2-(k-1)^2=2k-1$ from $k=1$ to $k=n$ and use the previous step to find:
\begin{enumerate}
\item[a.] a formula for $ \sum_{k=1}^n (2k-1)$.
\item[b.] a formula for $ \sum_{k=1}^n k$.
\end{enumerate}
\item Use the technique given in step 1, together with the results of step 2, to derive the formula for $\displaystyle \sum_{k=1}^n k^2$.\\
\textit{Hint: take $a_k=k^3$ in the telescoping sum in step 1.}
\end{enumerate}
\end{problem}

\begin{solution}\ \\
  \begin{enumerate}
    \item
    \begin{equation*}
        \begin{aligned}
         \sum_{j=1}^n (a_j-a_{j-1})& =a_1-a_0 + a_2-a_1 + a_3-a_2 + a_4-a_3\ldots + a_n - a_{n-1} + a_{n-1} - a_{n-2} + a_{n-2} - a_{n-3} \\
           & =\cancel{a_1}-a_0 + \cancel{a_2}-\cancel{a_1} + \cancel{a_3}-\cancel{a_2} + \cancel{a_4}-\cancel{a_3}\ldots + a_n - \cancel{a_{n-1}} + \cancel{a_{n-1}} - \cancel{a_{n-2}} + \cancel{a_{n-2}} - \cancel{a_{n-3}} \\
           &= a_n-a_0
       \end{aligned}
    \end{equation*}

     \item

      \begin{equation*}
        \begin{aligned}
          \sum_{k=1}^n (2k-1)  = &\sum_{k=1}^n k^{2} - (k-1)^{2} \\
          \text{Let } a_k = k^2 \\
          = &\sum_{k=1}^n a_k - (a_k-1)^{2} \\
          = & n^2 \\
          \sum_{k=1}^n (2k-1)  = & n^2 \\
          \sum_{k=1}^n 2k- \sum_{k=1}^n 1 &= n^2 \\
          2\sum_{k=1}^n k- n &= n^2 \\
          2\sum_{k=1}^n k &= n^2 +n \\
          \sum_{k=1}^n k &= \dfrac{n^2 +n}{2} \\
          \sum_{k=1}^n (2k-1)  = & \dfrac{n(n +1)}{2}
         \end{aligned}
      \end{equation*}

  \item
  \begin{equation*}
    \begin{aligned}
      \text{Let} a^k = k^3 \text{ so,}\\
      a_k - a_{k-1} &= k^3 - (k-1)^3 \\
      k^3 - (k-1)^3 &= k^3 - (k^3 - 1 -3k^2 +3k) \\
       &= 1 + 3k^2 -3k \\
       k^2 &= \dfrac{k^3 - (k-1)^3 + 3k -1}{3} \\
       \sum_{k=1}^n k^2 &= \dfrac{1}{3}\sum_{k=1}^n k^3 - (k-1)^3 + 3k -1 \\
       &= \dfrac{1}{3}(\sum_{k=1}^n k^3 - (k-1)^3 +\sum_{k=1}^n 3k -\sum_{k=1}^n1 ) \\
       \sum_{k=1}^n k^2 &= n^3 - 0 + 3\dfrac{n(n+1)}{2} - n \\
       \sum_{k=1}^n k^2 &= \dfrac{n(n+1)(2n+1)}{6}
    \end{aligned}
  \end{equation*}
\end{enumerate}
\end{solution}



\newpage

% (Recurrence Relations
\begin{problem} (25 points)\\
Solve the recurrence relation:
\begin{enumerate}
\item $A_n = 2\cdot A_{n-1} + 3 \text{, where } A_0 = 1$
\item $A_n = A_{n-1} + 4n - 2\text{, where } A_0 = 1$
\end{enumerate}
\end{problem}

\begin{solution}\ \\
  \begin{enumerate}
    \item
    \begin{equation*}
      \begin{aligned}
            A_n &= 2\cdot A_{n-1} + 3 \text{, where } A_0 = 1 \\
            A_n &= 2A_{n-1} + 3 \\
        A_{n-1} &= 2A_{n-2} + 3 \\
        A_{n-2} &= 2A_{n-3} + 3 \\
            A_n &= 2(2(2A_{n-3} + 3) + 3) + 3 \\
            A_n &= 2^{3}A_{n-3} + 3(3) \\
            A_n &= 2^{n}A_{0} + 3n \text{ Since $A_0 = 1$} \\
            A_n &= 2^{n} + 3n
      \end{aligned}
    \end{equation*}

  \item
  \begin{equation*}
    \begin{aligned}
      A_n = A_{n-1} + 4n - 2\text{, where } A_0 = 1 \\
    \end{aligned}
  \end{equation*}
\end{enumerate}
\end{solution}

\newpage

% Languages and Grammars
\begin{problem} (25 points)\\
Let $V=\{S,A,B,a,b\}$ and $T=\{a,b\}$.  Find the language generated by the grammar $(V,T,S,P)$ when the set $P$ of production rules consists of
\begin{enumerate}
  \item $S\to AB$, $A \to ab$, $B\to bb$.
  \item $S \to AB$, $S \to aA$, $A \to a$, $B \to ba$.
  \item $S \to AB$, $S \to AA$, $A\to aB$, $A\to ab$, $B\to b$.
  \item $S \to AA$, $S \to B$, $A \to aaA$, $A \to aa$, $B \to bB$, $B \to b$.
  \item $S \to AB$, $A \to aAb$, $B \to bBa$, $A \to \lambda$, $B \to \lambda$.
\end{enumerate}
\end{problem}

\begin{solution}\ \\
  \begin{enumerate}
  \item $S\to AB$, $A \to ab$, $B\to bb$. \\
    From $S$ we get $AB$ \\
    From $A$ we now have $abB$ \\
    From $B$ we now have $aabb$ \\
    $\therefore$ the language is $\{aabb\}$

  \item $S \to AB$, $S \to aA$, $A \to a$, $B \to ba$. \\
    From $S$ we get $AB$ \\
    From $A$ we now have $aB$ \\
    From $B$ we now have $\{aba\}$ \\
    Also, from $S$ we get $aA$ \\
    From $A$ we now have $\{aa\}$ \\
    $\therefore$ the language is $\{aba, aa\}$

  \item $S \to AB$, $S \to AA$, $A\to aB$, $A\to ab$, $B\to b$.
    From $S$ we get $AB$ \\
    From $A$ we now have $aBB$ \\
    Using $B$ twice we get $abb$ \\
    From $S$ we get $AA$ \\
    From $A$ we now have $aBaB$ \\
    Using $B$ twice we get $abab$ \\
    $\therefore$ the language is $\{abb, abab\}$

  \item $S \to AA$, $S \to B$, $A \to aaA$, $A \to aa$, $B \to bB$, $B \to b$. \\
    From $S$ we get $AA$ \\
    From both $A$'s we forms of $A$ such that $aaaa$ or $aaaaa$ or $aaaaaa$ or $aaaaaaaa$, strings of even $a$'s with minimum size of 4 \\
    Using $S \to B$ to dervive B we get forms of $b$, $bb$, $bbb$, such that it results in strings of $b$ greater than $1$\\
    $\therefore$ the language is $\{a\cdot 2^{n} \mid n \geq 2 \} \cup \{b^2 \mid n \geq 1 \}$

  \item $S \to AB$, $A \to aAb$, $B \to bBa$, $A \to \lambda$, $B \to \lambda$
    From $S$ we get $AB$ \\
    From $A$ we get $aBa$ all the way to $ababababab\ldots AB$ until $\lambda$ depending on $x$ number of repetitions\\
    From $B$ we get $abbaAB$ all the way to $ababababab\ldots AB$ until $\lambda$ depending on $y$ number of repetitions\\
    $\therefore$ the language is of the form $\{a^{x+y}b^{x+y} \mid x \geq 0, y \geq 0\}$
  \end{enumerate}
\end{solution}

\newpage

% Languages and Grammars
\begin{problem} (25 points)\\
Find a phrase-structure grammar for each of these languages.
\begin{enumerate}
\item the set consisting of the bitstrings 00, 11, and 010.
\item the set of bitstrings that start with 10 and end with one or more 1s.
\item the set of bitstrings consisting of an odd number of 0s followed by a final 1.
\item the set of bitstrings that have neither two consecutive 0s nor two consecutive 1s.
\end{enumerate}
\end{problem}

\begin{solution}\ \\
  Let $V=\{S,A,B,a,b\}$ and $T=\{a,b\}$.  Find the language generated by the grammar $(V,T,S,P)$ when the set $P$ of production rules consists of
  \begin{enumerate}
  \item the set consisting of the bitstrings 00, 11, and 010.
    With $L = \{ 00, 11, 010\}$ \\
    The phrase structure gammer of this language is $G = \{ V,T,S,P\}$ \\
    The Vocabulary $(V) = \{0,1,S\}$ and the terminal symbols are $T = \{ 0,1\}$ \\
    The productions are $S \to 00, S \to  11$, and $S \to 010$

  \item the set of bitstrings that start with 10 and end with one or more 1s. \\
    The language $L = \{ a: a$ is a bit string that starts with $10$ and end with one or more $1$s$\}$ \\
    The phrase structure gammer of this language is $G = \{ V,T,S,P\}$ \\
    The Vocabulary $(V) = \{0,1,S, A, B\}$ and the terminal symbols are $T = \{ 0,1\}$ \\
    The productions are $S \to 10AB, A \to  AA, A \to 1, A \to 0, B \to BB, B \to 1$

  \item the set of bitstrings consisting of an odd number of 0s followed by a final 1. \\
    The language $L = \{ a: a$ is a bit string of odd number of $0$s followed by a final $1\}$ \\
    The phrase structure gammer of this language is $G = \{ V,T,S,P\}$ \\
    The Vocabulary $(V) = \{0,1,S, A, B, C\}$ and the terminal symbols are $T = \{ 0,1\}$ \\
    The productions are $S \to A1, A \to  \lambda, A \to BBC, A \to BCB, A \to CBB, B \to CB, B \to BC, B \to 1 , C \to 0$

  \item the set of bitstrings that have neither two consecutive 0s nor two consecutive 1s. \\
  The language $L = \{ a: a$ is a bit string that  has neither two consecutive $0$s nore two consecutive $1$s$\}$ \\
  The phrase structure gammer of this language is $G = \{ V,T,S,P\}$ \\
  The Vocabulary $(V) = \{0,1,S, A, B\}$ and the terminal symbols are $T = \{ 0,1\}$ \\
  The productions are $S \to A, A \to  AA, A \to 01, A \to \lambda, S \to B, B \to BB, B \to 10, B \to \lambda$
  \end{enumerate}
\end{solution}
\bigskip
\honor

\bigskip
\checklist
\end{document}
