\documentclass{article}
\usepackage{amsmath,amssymb,amsthm,latexsym,paralist,url}
\usepackage[margin=1in]{geometry}
\usepackage{cancel}

\theoremstyle{definition}
\newtheorem{problem}{Problem}
\newtheorem*{solution}{Solution}
\newtheorem*{resources}{Resources}


\newcommand{\honor}{\noindent \textbf{Aggie Honor Statement: }On my honor as an Aggie, I have neither
  given nor received any unauthorized aid on any portion of the academic work included in this assignment.
}


\newcommand{\checklist}{\noindent\textbf{Checklist:}
Did you...
\begin{compactenum}
\item abide by the Aggie Honor Code?
\item solve all problems?
\item start a new page for each problem?
\item show your work clearly?
\item type your solution?
\item submit a PDF to eCampus?
\end{compactenum}
}

\newcommand{\problemset}[1]{\begin{center}\textbf{Problem Set #1}\end{center}}
\newcommand{\duedate}[1]{\begin{quote}\textbf{Due: #1} on eCampus (\url{ecampus.tamu.edu}). \\You must show your work in order to recieve credit.\end{quote}}
\newcommand{\mysectionnumber}[0]{503}

\title{CSCE 222: Discrete Structures for Computing\\Section \mysectionnumber\\Fall 2016}
\author{YOUR NAME HERE}

\begin{document}

\maketitle

\problemset{6}

\duedate{9 October 2016 (Sunday) before 11:59 p.m.}

\bigskip

% Sequences and Sums
\begin{problem} (25 points)
\begin{enumerate}
\item Show that $ \sum_{j=1}^n (a_j-a_{j-1})=a_n-a_0$, where $a_0,a_1,\ldots,a_n$ is a sequence of real numbers.\\ \textit{This type of sum is called \textbf{telescoping}.}
\item Sum both sides of the identity $k^2-(k-1)^2=2k-1$ from $k=1$ to $k=n$ and use the previous step to find:
\begin{enumerate}
\item[a.] a formula for $ \sum_{k=1}^n (2k-1)$.
\item[b.] a formula for $ \sum_{k=1}^n k$.
\end{enumerate}
\item Use the technique given in step 1, together with the results of step 2, to derive the formula for $\displaystyle \sum_{k=1}^n k^2$.\\
\textit{Hint: take $a_k=k^3$ in the telescoping sum in step 1.}
\end{enumerate}
\end{problem}

\begin{solution}\ \\
  \begin{enumerate}
    \item \ \\
     $\sum_{j=1}^n (a_j-a_{j-1})=a_1-a_0 + a_2-a_1 + a_3-a_2 + a_4-a_3\ldots + a_n - a_{n-1} + a_{n-1} - a_{n-2} + a_{n-2} - a_{n-3}$


     $\sum_{j=1}^n (a_j-a_{j-1})=\cancel{a_1}-a_0 + \cancel{a_2}-\cancel{a_1} + \cancel{a_3}-\cancel{a_2} + \cancel{a_4}-\cancel{a_3}\ldots + a_n - \cancel{a_{n-1}} + \cancel{a_{n-1}} - \cancel{a_{n-2}} + \cancel{a_{n-2}} - \cancel{a_{n-3}}$


     $= a_n-a_0$

     \item sum



  \end{enumerate}
\end{solution}

\newpage

% (Recurrence Relations
\begin{problem} (25 points)\\
Solve the recurrence relation:
\begin{enumerate}
\item $A_n = 2\cdot A_{n-1} + 3 \text{, where } A_0 = 1$
\item $A_n = A_{n-1} + 4n - 2\text{, where } A_0 = 1$
\end{enumerate}
\end{problem}

\begin{solution}\ \\

\end{solution}

\newpage

% Languages and Grammars
\begin{problem} (25 points)\\
Let $V=\{S,A,B,a,b\}$ and $T=\{a,b\}$.  Find the language generated by the grammar $(V,T,S,P)$ when the set $P$ of production rules consists of
\begin{enumerate}
\item $S\to AB$, $A \to ab$, $B\to bb$.
\item $S \to AB$, $S \to aA$, $A \to a$, $B \to ba$.
\item $S \to AB$, $S \to AA$, $A\to aB$, $A\to ab$, $B\to b$.
\item $S \to AA$, $S \to B$, $A \to aaA$, $A \to aa$, $B \to bB$, $B \to b$.
\item $S \to AB$, $A \to aAb$, $B \to bBa$, $A \to \lambda$, $B \to \lambda$.
\end{enumerate}
\end{problem}

\begin{solution}\ \\

\end{solution}

\newpage

% Languages and Grammars
\begin{problem} (25 points)\\
Find a phrase-structure grammar for each of these languages.
\begin{enumerate}
\item the set consisting of the bitstrings 00, 11, and 010.
\item the set of bitstrings that start with 10 and end with one or more 1s.
\item the set of bitstrings consisting of an odd number of 0s followed by a final 1.
\item the set of bitstrings that have neither two consecutive 0s nor two consecutive 1s.
\end{enumerate}
\end{problem}

%\begin{solution}\ \\
%\end{solution}


\bigskip
\honor

\bigskip
\checklist
\end{document}
