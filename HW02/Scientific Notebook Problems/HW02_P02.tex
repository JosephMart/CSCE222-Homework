
\documentclass{article}
%%%%%%%%%%%%%%%%%%%%%%%%%%%%%%%%%%%%%%%%%%%%%%%%%%%%%%%%%%%%%%%%%%%%%%%%%%%%%%%%%%%%%%%%%%%%%%%%%%%%%%%%%%%%%%%%%%%%%%%%%%%%%%%%%%%%%%%%%%%%%%%%%%%%%%%%%%%%%%%%%%%%%%%%%%%%%%%%%%%%%%%%%%%%%%%%%%%%%%%%%%%%%%%%%%%%%%%%%%%%%%%%%%%%%%%%%%%%%%%%%%%%%%%%%%%%
%TCIDATA{OutputFilter=LATEX.DLL}
%TCIDATA{Version=5.50.0.2960}
%TCIDATA{<META NAME="SaveForMode" CONTENT="1">}
%TCIDATA{BibliographyScheme=Manual}
%TCIDATA{Created=Friday, September 09, 2016 00:44:10}
%TCIDATA{LastRevised=Friday, September 09, 2016 01:46:26}
%TCIDATA{<META NAME="GraphicsSave" CONTENT="32">}
%TCIDATA{<META NAME="DocumentShell" CONTENT="General\Blank Document">}
%TCIDATA{CSTFile=Math with theorems suppressed.cst}
%TCIDATA{PageSetup=72,72,72,72,0}
%TCIDATA{AllPages=
%F=36,\PARA{038<p type="texpara" tag="Body Text" >\hfill \thepage}
%}


\newtheorem{theorem}{Theorem}
\newtheorem{acknowledgement}[theorem]{Acknowledgement}
\newtheorem{algorithm}[theorem]{Algorithm}
\newtheorem{axiom}[theorem]{Axiom}
\newtheorem{case}[theorem]{Case}
\newtheorem{claim}[theorem]{Claim}
\newtheorem{conclusion}[theorem]{Conclusion}
\newtheorem{condition}[theorem]{Condition}
\newtheorem{conjecture}[theorem]{Conjecture}
\newtheorem{corollary}[theorem]{Corollary}
\newtheorem{criterion}[theorem]{Criterion}
\newtheorem{definition}[theorem]{Definition}
\newtheorem{example}[theorem]{Example}
\newtheorem{exercise}[theorem]{Exercise}
\newtheorem{lemma}[theorem]{Lemma}
\newtheorem{notation}[theorem]{Notation}
\newtheorem{problem}[theorem]{Problem}
\newtheorem{proposition}[theorem]{Proposition}
\newtheorem{remark}[theorem]{Remark}
\newtheorem{solution}[theorem]{Solution}
\newtheorem{summary}[theorem]{Summary}
\newenvironment{proof}[1][Proof]{\noindent\textbf{#1.} }{\ \rule{0.5em}{0.5em}}
\input{tcilatex}
\begin{document}


\begin{problem}
(20 points)\newline
Do Supplementary Exercise 29 of Chapter 3 (page 234).

a) Use pseudocode to specify a brute-force algorithm that determines when
given as input a sequence of n positive integers whether there are two
distinct terms of the sequence that have as sum a third term. The algorithm
should loop through all triples of terms of the sequence, checking whether
the sum of the first two terms equals the third. b) Give a big-O estimate
for the complexity of the bruteforce algorithm from part (a).
\end{problem}

\begin{solution}
\end{solution}

\qquad \qquad \qquad \textbf{procedure }\textit{brute-force(}$%
a_{1},a_{2},a_{3},...,a_{n}$)

\qquad \qquad \textbf{\qquad for }$i=1$\textbf{\ to }$n\qquad \qquad \qquad
\qquad \qquad \qquad O(n)$

\textbf{\qquad \qquad \qquad \qquad for }$j=i+1$\textbf{\ to }$n\qquad
\qquad \qquad \qquad O(n)$

\textbf{\qquad \qquad \qquad \qquad \qquad for }$k=1$\textbf{\ to }$n\qquad
\qquad \qquad \qquad O(n)$

\textbf{\qquad \qquad \qquad \qquad \qquad \qquad if }$a_{i}+a_{j}=a_{k}$%
\textbf{\ then\qquad \qquad }$O(1)$

\textbf{\qquad \qquad \qquad \qquad \qquad \qquad \qquad return }\textit{true%
}

\textbf{\qquad \qquad \qquad \qquad \qquad \qquad else}

\textbf{\qquad \qquad \qquad \qquad \qquad \qquad return }\textit{false}

\qquad \qquad \qquad The complexity fo the algorithm is $O(n^{3})$

\end{document}
