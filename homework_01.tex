\documentclass{article}
\usepackage{amsmath,amsthm,latexsym,paralist,url}
\usepackage[margin=1in]{geometry}

\theoremstyle{definition}
\newtheorem{problem}{Problem}
\newtheorem*{solution}{Solution}
\newtheorem*{resources}{Resources}


\newcommand{\honor}{\noindent \textbf{Aggie Honor Statement: }On my honor as an Aggie, I have neither
  given nor received any unauthorized aid on any portion of the academic work included in this assignment.
}

 
\newcommand{\checklist}{\noindent\textbf{Checklist:}
\begin{compactenum}
\item Did you abide by the Aggie Honor Code?
\item Did you solve all problems and start a new page for each? 
\item Did you submit the PDF to eCampus?
\end{compactenum}
}

\newcommand{\problemset}[1]{\begin{center}\textbf{Problem Set #1}\end{center}}
\newcommand{\duedate}[1]{\begin{quote}\textbf{Due: #1} on eCampus (\url{ecampus.tamu.edu}).\end{quote}}
\newcommand{\mysectionnumber}[0]{503}

\title{CSCE 222: Discrete Structures for Computing\\Section 503\\Fall 2016}
\author{Joseph Martinsen}

\begin{document}

\maketitle

\problemset{1}

\duedate{4 September 2016 (Sunday) before 11:59 p.m.}

\bigskip

% Algorithms
\begin{problem} (20 points)\\
You have $n$ coins, exactly one of which is counterfeit.  You know counterfeit coins weigh more than authentic coins.  Devise an algorithm for finding the counterfeit coin using a balance scale\footnote{A balance scale compares the weight of objects placed on it.  The result of the comparison is either left side heavier, right side heavier, or both sides equal.}.  Express your algorithm in pseudocode.  For $n=12$, how many weighings does your algorithm use? 
\end{problem}

%\begin{solution}
%\end{solution}

\newpage

% Algorithms
\begin{problem} (20 points)\\
Devise an algorithm that takes as input a list of $n$ integers in unsorted order, where the integers are not necessarily distinct, and outputs the location (index of first element) and length of the longest contiguous non-decreasing subsequence in the list.  If there is a tie, it outputs the location of the first such subsequence.  Express your algorithm in pseudocode.  For the list $9,7,9,4,5,8,1,0,5,9$, what is the algorithm's output? How many comparison operations between elements of the list are used?
\end{problem}

\begin{solution}
\item
\textbf{procedure} \textit{non-decreasing-location}(\textit{$a_1, a_2, a_3,...,a_n$}: integers)\newline
$location:=1$\newline
\textbf{for} i:=$2$ \textbf{to} $n-1$\par
	\textbf{if} $a_{i-1} < a_i$ \textbf{then}\par
    	$location:=i$\newline
        \textbf{else if} $a_{i-1} = a_i$ \textbf{then}\newline
        \textbf{return} \textit{non-decreasing-location}\{\textit{non-decreasing-location} is the location of the first element in the subsequence of non-decreacing elements\}

\end{solution}

\newpage

% Growth of Functions - Big O
\begin{problem} (20 points)\\
Arrange the following functions in order such that each function is big-$O$ of the next function:  $2\cdot3^n$, $3n!$, $2019\log{n}$, $\displaystyle \frac{n^3}{10^6}$, $n\log{n}$,  $\sqrt{n}$, $3\cdot2^n$.  Prove your answer is correct by giving the witnesses for each pair of consecutive functions.
\end{problem}

\begin{solution}
\item The following functions are arranged such that that each function is a big-$O$ of the next function.
\begin{enumerate}
\item $2019\log{n}$ \newline
$2019\log{n}$ is $O(\sqrt{n})$ by taking $C = 2019$ and $k = 1$ as witnesses. 
\item $\sqrt{n}$
\item $n\log{n}$
\item $3\cdot2^n$
\item $2\cdot3^n$
\item $3n!$
\bigskip
\end{enumerate}

$\sqrt{n}$ is $O(n\log{n})$ by taking  $C = 1$ and $k = 4$ as witnesses. \bigskip\newline
$n\log{n}$ is $O(3\cdot2^n)$ by taking  $C = 1$ and $k = 1$ as witnesses. \bigskip\newline
$3\cdot2^n$ is $O(2\cdot3^n)$ by taking  $C = \dfrac{3}{2}$ and $k = 0$ as witnesses. \bigskip\newline
$2\cdot3^n$ is $O(3n!)$ by taking  $C = \dfrac{2}{3}$ and $k = 7$ as witnesses. \bigskip\newline
\end{solution}

\newpage

% Growth of Functions - Big O
\begin{problem} (20 points)\\
For each of the following functions, give a big-$O$ estimate, including witnesses, using a simple function $g(n)$ of the smallest order:
\begin{enumerate}
\item $(n^2+8)(n+1)$
\item $(n\log{n} + 1)^2+(\log{n}+1)(n^2+1)$
\item $\displaystyle n^{2^n}+n^{n^2}$
\item $\displaystyle \frac{n^4+5\log{n}}{x^3+1}$
\item $2x^4+7x^3+5x+3$
\end{enumerate}
\end{problem}

%\begin{solution}
%\end{solution}

\newpage

% Growth of Functions - Big Theta
\begin{problem} (20 points)\\
For each of the following functions, determine whether that function is of the same order as $n^2$ either by finding witnesses or showing that sufficient witnesses do not exist:
\begin{enumerate}
\item $13n+12$
\item $n^2+1000 n\log{n}$
\item $3^n$
\item $3n^2+n-5$
\item $\displaystyle \frac{n^3+2n^2-n+3}{4n}$
\end{enumerate}
\end{problem}



\begin{solution}

\item Suppose that there are constants $C$ and $k$ for which $n^2 \leq C(13n+12)$ whenever $n > k$. We can divide both 
		sides of $Cn^2 \leq 13n+12$ by $n$ and simplify to $n \leq 3C$ for values of $n > k$. However, no matter what 
        $C$ and $k$ are, the inequality $n \leq 3C$ cannot hold for all $n$ with $n > k$. In particular, once we set a 
        value of $k$, we see that when $n$ is larger than the maximum of $k$ and $C$, it is not true that $n \leq C$ 
        even though $n > k$. This contradiction shows that $n^2$ is not $O(13n+12)$. Thus the function $13n+12$ is not 
        as the same order as $n^2$.
\item Suppose that there are particular witnesses $C$ and $k$ for which $n^2+1000n\log{n} \leq Cn^2$. After dividing both sides by $n$ the resultant is $ n + \log{{n}^{1000}}\leq Cn$.
\item $3^n$
\item $3n^2+n-5$
\item $\displaystyle \frac{n^3+2n^2-n+3}{4n}$

\end{solution}

\newpage


\bigskip
\honor

\bigskip
\checklist
\end{document}
