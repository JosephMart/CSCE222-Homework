\documentclass{article}
\usepackage{amsmath,amssymb,amsthm,latexsym,paralist,url}
\usepackage[margin=1in]{geometry}
\usepackage{tikz}
\usepackage[procnumbered, ruled, linesnumbered, commentsnumbered, noend]{algorithm2e}
\usetikzlibrary{arrows,automata}

\theoremstyle{definition}
\newtheorem{problem}{Problem}
\newtheorem*{solution}{Solution}
\newtheorem*{resources}{Resources}


\newcommand{\honor}{\noindent \textbf{Aggie Honor Statement: }On my honor as an Aggie, I have neither
  given nor received any unauthorized aid on any portion of the academic work included in this assignment.
}

 
\newcommand{\checklist}{\noindent\textbf{Checklist:}
Did you...
\begin{compactenum}
\item abide by the Aggie Honor Code?
\item solve all problems?
\item start a new page for each problem?
\item show your work clearly?
\item type your solution?
\item submit a PDF to eCampus?
\end{compactenum}
}

\newcommand{\problemset}[1]{\begin{center}\textbf{Problem Set #1}\end{center}}
\newcommand{\duedate}[1]{\begin{quote}\textbf{Due: #1} on eCampus (\url{ecampus.tamu.edu}). \\You must show your work in order to recieve credit.\end{quote}}
\newcommand{\mysectionnumber}[0]{503}

\title{CSCE 222: Discrete Structures for Computing\\Section \mysectionnumber\\Fall 2016}
\author{Joseph Martinsen}

\begin{document}

\maketitle

\problemset{11}

\duedate{13 November 2016 (Sunday) before 11:59 p.m.}

\bigskip

% Recursive Definitions
\begin{problem} (25 points)
\begin{enumerate}
\item Give a recursive definition for the set of bitstrings that have more 0s than 1s.
\item Give a recursive definition for the set of bitstrings that have twice as many 0s as 1s.
\end{enumerate}

\end{problem}

\begin{solution}\ \\
\begin{enumerate}
  \item 
  \begin{align*}
    0 &\in S \\
    x,y \in S &\rightarrow xy1, x1y, 1xy,xy \in S
  \end{align*}
  \item
    \begin{align*}
    0 &\in S \\
    x,y \in S &\rightarrow 1x00, 00x1, 0x1y0,0xy \in S
  \end{align*}
\end{enumerate}
\end{solution}

\newpage

% Structural Induction
\begin{problem} (25 points)\\
The \textbf{reversal} of a string $w$, denoted $w^R$, is the string consisting of the symbols of $w$ in reverse order.
\begin{enumerate}
\item Give a recursive definition for the reversal of a string. \textit{Hint: First define the reversal of the empty string.  Then write a string $w$ of length $n+1$ as $xy$, where $x$ is a string of length $n$ and $y\in\Sigma$, and express the reversal of $w$ in terms of $x^R$ and $y$.}
\item Use structural induction to prove that $(w_1w_2)^R = w_2^Rw_1^R$.
\end{enumerate}
\end{problem}

\begin{solution}\ \\
\end{solution}

\newpage

% Structural Induction
\begin{problem} (25 points)\\
The set of leaves and the set of internal vertices of a full binary tree can be defined recursively.\\
\\
\textit{Basis Step:} The root $r$ is a leaf of the full binary tree with exactly one vertex $r$.  This tree has no internal vertices.\\
\\
\textit{Recursive Step:} The set of leaves of the tree $T=T_1\cdot T_2$ is the union of the sets of leaves of $T_1$ and of $T_2$.  The internal vertices of $T$ are the root $r$ of $T$ and the union of the sets of internal vertices of $T_1$ and of $T_2$.\\
\\
Use structural induction to prove that $\ell(T)$, the number of leaves of a full binary tree $T$, is 1 more than $i(T)$, the number of internal vertices of $T$.
\end{problem}

\begin{solution}\ \\
\end{solution}

\newpage

% Recursive Algorithms
\begin{problem} (25 points)
\begin{enumerate}
\item Give a recursive algorithm for finding the reversal of a string.
\item Prove that your recursive algorithm is correct.
\end{enumerate}
\end{problem}

\begin{solution}\ \\
\begin{algorithm}
\DontPrintSemicolon
\caption{reversal( x: bit string)}
$l := length(x)$\;
\If{$l\leq 1$}{
	$reversal(x) := x$\;
}
\Else{
	$reversal(x) := substr(x,l,l)reversal(substr(x,1,l-1)$\;
}
\end{algorithm}
substr($a$,$b$,$c$) is the substring of $a$ consisting of the symbols in the $b$ through $c$ position
\end{solution}

\newpage


\bigskip
\honor

\bigskip
\checklist
\end{document}
