\documentclass{article}
\usepackage{amsmath,amsthm,latexsym,paralist,url}
\usepackage{amssymb}
\usepackage[margin=1in]{geometry}

\theoremstyle{definition}
\newtheorem{problem}{Problem}
\newtheorem*{solution}{Solution}
\newtheorem*{resources}{Resources}


\newcommand{\honor}{\noindent \textbf{Aggie Honor Statement: }On my honor as an Aggie, I have neither
  given nor received any unauthorized aid on any portion of the academic work included in this assignment.
}


\newcommand{\checklist}{\noindent\textbf{Checklist:}
\begin{compactenum}
\item Did you abide by the Aggie Honor Code?
\item Did you solve all problems and start a new page for each?
\item Did you show your work clearly?
\item Did you submit the PDF to eCampus?
\end{compactenum}
}

\newcommand{\problemset}[1]{\begin{center}\textbf{Problem Set #1}\end{center}}
\newcommand{\duedate}[1]{\begin{quote}\textbf{Due: #1} on eCampus (\url{ecampus.tamu.edu}). You must show your work in order to recieve credit.\end{quote}}
\newcommand{\mysectionnumber}[0]{503}

\title{CSCE 222: Discrete Structures for Computing\\Section \mysectionnumber\\Fall 2016}
\author{Joseph Martinsen}

\begin{document}

\maketitle

\problemset{3}

\duedate{18 September 2016 (Sunday) before 11:59 p.m.}

\bigskip

% Predicates and Quantifiers
\begin{problem} (20 points)\\
Excercise 60 of Section 1.4 (page 56).
\end{problem}

\begin{solution}\ \\
Universal Discourse of x consists of all English text
P(x): x is a clear explanation\\
Q(x): x is satisfactory\\
R(x): x is an excuse\\
\begin{enumerate}[a)] % a), b), c), ...
\item %a)
$\forall x(P(x) \rightarrow Q(x))$

\item
$\exists x(R(x) \wedge \neg Q(x))$

\item
$\exists x(R(x) \wedge \neg P(x))$

\item
\begin{tabular}{ll}
Step & Reason \\
1. $\exists x(R(x) \wedge \neg Q(x))$ & Premise \\
2. $R(a) \wedge \neg Q(a)$            & Existential instantiation on \bf{1} \\
3. $\forall x(P(x) \rightarrow Q(x))$ & Premise \\
4. $P(a) \rightarrow Q(x)$            & Universal instantiation on \bf{3} \\
5. $\neg Q(x)$                        & By simplification on \bf{2} \\
6. $\neg P(x)$                        & Modus Tollens of \textbf{4} and \bf{5} \\
7. $R(a)$                             & By simplification of \bf{2} \\
8. $R(a) \wedge \neg P(a)$            & Conjunction of \textbf{6} and \bf{7} \\
9. $\exists (R(x) \wedge \neg P(x))$  & Existential Generalization of \bf{8}
\end{tabular} \\ \ \\ \ \\
Holding \textbf{a)} and \textbf{b)} as valid premises, \textbf{c)} does follow.

\end{enumerate}



\end{solution}

\newpage

% Nested Quantifiers
\begin{problem} (20 points)\\
Exercise 36 of Section 1.5 (page 68).
\end{problem}

\begin{solution}\ \\
\begin{enumerate}[a)]
\item No one has lost more than \$1000 playing the lottery \\
      Let $L(x,y)$ represent a person $x$ who has lost $y$ dollars playing the lottery \\
      The statement using quantifiers is:

      $\neg \exists x \exists y ( y > 1000 \wedge L(x,y) )$

      Now negate the statement

      $\neg[\neg \exists x \exists y ( y > 1000 \wedge L(x,y) )] \equiv \exists x \forall y
          \neg (y > 1000 \wedge L(x,y) )$

      This reads as \textbf{someone has lost less than or equal to \$1000 playing the lottery}

\item There is a student in this class who has chatted with exactly one other student \\
      Let $B(x,y)$ represent a student in class, $x$, has chatted with a student in class $y$ but not any of the
      other students in class, $z$

      $\exists x \exists y \forall z ( x \neq y \wedge x \neq z \wedge B(x,y) \wedge \neg B(x,z) )$

      Now Negate the statement:

      $\neg [\exists x \exists y \forall z ( x \neq y \wedge x \neq z \wedge B(x,y) \wedge \neg B(x,z) )] \equiv
          \forall x \forall y \exists z \neg ( x \neq y \wedge x \neq z \wedge B(x,y) \wedge \neg B(x,z) )$

      This reads as \textbf{everyone in class has spoken with no one or everyone}

\item No student in this class has sent e-mail to exactly two other students in class \\
      Let $A(x,y)$ represent a student in class, $x$ who has sent an email to student in class $y$

      $\neg \exists x \exists y \exists z ( y \neq z \wedge x \neq z \wedge ( A(x,y) \wedge A(x,z)  )  )$

      Now Negate the statement:

      $\neg [\neg \exists x \exists y \exists z ( y \neq z \wedge x \neq z \wedge ( A(x,y) \wedge A(x,z))) ] \equiv
          \exists x \forall y \forall z \neg(y \neq z \wedge x \neq z \wedge ( A(x,y) \wedge A(x,z)))$

      This reads as \textbf{some student in class has sent email to exactly two other students in class}

\item Some student has solved every exercise in this book \\
      Let $D(x,y)$ represents a student in class $x$ who has solved exercise $y$ in the book

      $\exists x \forall y D(x,y)$

      Now negate the statement:

      $\neg [\exists x \forall y D(x,y)] \equiv \forall \exists y D(x,y)$

      This reads as \textbf{every student in class has solved an exercise in the book}

\item No student has solved at least one exercise in every section of the book \\
      Let $P(x,y)$ represent a student $x$ in class has solved exercise $y$ \\
      Let $B(y,z)$ represent a exercise $y$ in section $z$ of the book

      $\neg \exists x \exists y \forall z ( P(x,y) \wedge B(y,z) )  $

      Now negate it:

      $\neg [ \neg \exists x \exists y \forall z ( P(x,y) \wedge B(y,z) ]) \equiv
          \exists x \exists y \forall z ( P(x,y) \wedge B(y,z)$

      This reads as \textbf{some student has solved at least one exercise in every section of this book}

\end{enumerate}
\end{solution}

\newpage

% Rules of Inference
\begin{problem} (20 points)\\
Exercise 34 of Section 1.6 (page 80).
\end{problem}

\begin{solution}\ \\
1. Logic is difficult or not many like logic \\
2. If mathematics is easy then logic is not difficult \\
The universal discourse for $x$ is all students \\
$A(x)$ represents logic is difficult to students \\
$B(x)$ represents many students like logic \\
$C(x)$ represents mathematics is easy to students \\
The premises can be rewritten as \\
1. $A(x) \vee \neg B(x)$ \\
2. $C(x) \rightarrow \neg A(x)$
\begin{enumerate}[a)]
\item That mathematics is not easy, if many students like logic

      $B(x) \rightarrow \neg C(x) \equiv \neg B(x) \vee \neg C(x)$

      The argument is:

      $A(x) \vee \neg B(x)$ \\
      $\displaystyle \dfrac{\neg C(x) \rightarrow \neg A(x)}{\therefore \neg B(x) \vee \neg C(x)}$

      Using rules of inference resolution, this conclusion is valid \\
      Since $\neg B(x) \vee \neg C(x)$ is valid thus $B(x) \rightarrow \neg C(x)$ is valid as well

\item That not many students like logic, if mathematics is not easy

      $\neg C(x) \rightarrow \neg B(x) \equiv C(x) \vee \neg B(x)$

      The argument is:

      $A(x) \vee \neg B(x)$ \\
      $\displaystyle \dfrac{\neg C(x) \rightarrow \neg A(x)}{\therefore C(x) \vee \neg B(x)}$

      If the inference is used, by resolution, the correct conclusion would be $\neg C(x) \vee \neg B(x)$ not $C(x)
      \vee \neg B(x) \therefore$ the statement \textbf{That not many students like logic, if mathematics is not easy}
      is \textbf{invalid}

\item That mathematics is not easy or logic is difficult.

      $\neg C(x) \vee A(x)$

      The argument is:

      $A(x) \vee \neg B(x)$ \\
      $\displaystyle \dfrac{\neg C(x) \rightarrow \neg A(x)}{\therefore \neg C(x) \vee A(x)}$

      If inference by resolution is used, the correct conclusion would be $ \neg B(x) \vee \neg C(x)$ not $\neg C(x)
      \vee A(x) \therefore$ the statement \textbf{That mathematics is not easy or logic is difficult} is
      \textbf{invalid}

\item That logic is not difficult or mathematics is not easy.

      $\neg A(x) \vee \neg C(x)$

      $A(x) \vee \neg B(x)$ \\
      $\displaystyle \dfrac{\neg C(x) \rightarrow \neg A(x)}{\therefore \neg A(x) \vee \neg C(x)}$

      This statement is $\equiv$ $C(x) \rightarrow \neg A(x)$ which is exactly the same as the $2^{nd}$ premise
      $\therefore$ the statement $\neg A(x) \vee \neg C(x)$ is \textbf{valid}

\item That if not many students like logic, then either mathematics is not easy or logic is not difficult.

      \begin{tabular}{ll}
        $\neg B(x) \rightarrow ( \neg C(x) \vee \neg A(x) )$ & Initial statement \\
        $\neg B(x) \rightarrow \neg ( C(x) \wedge A(x))$     & By DeMorgan's Law \\
        $\neg(\neg B(x)) \vee \neg ( C(x) \wedge A(x))$      & By $p \rightarrow q \equiv \neg p \vee q$ \\
        $\neg ( \neg B(x) \wedge (C(x) \wedge A(x))$         & By DeMorgans's Law
      \end{tabular} \\ \ \\
      Since $D(x)$ and $\neg D(x)$ both appear in the premise, in order for $\neg ( \neg B(x) \wedge (M(x) \wedge
      A(x))$ to be valid, $\neg B(x)$ is false or $ \neg C(x)$ is true. This condition follows $\neg B(x) \vee \neg
      C(x)$ which was proved to be valid in \textbf{a)} $\therefore$ $\neg B(x) \rightarrow ( \neg C(x) \vee \neg
      A(x) )$ is \textbf{valid}

\end{enumerate}
\end{solution}

\newpage

% Introduction to Proofs
\begin{problem} (20 points)\\
Exercises 18 and 30 of Section 1.7 (page 91).
\end{problem}

\begin{solution}\ \\
\textbf{18.}
\begin{enumerate}[a)]
\item Proof by contraposition \\
      Assume that $n$ is odd \\
      Then $n = 2k +1$ using some integer $k$ \\
      $3n+2=3(2k+1)+2$ \\
      $=6k + 3 + 2$ \\
      $=6k + 5$ \\
      $=6k + 4 + 1 $ \\
      $= 2(3k+2) +1$ is odd $\therefore$ $3n+2$ is odd thus if $n$ is an integer and $3n+2$ is even then $n$ is even

\item Proof by contradiction \\ \ \\
      Suppose $3n+2$ is even and $n$ is odd \\ \ \\
      Let n and m be any two odd integers. Using definition of odd we have that $n = 2a + 1$ and $m = 2b + 1$.
      Multiplying the two together, the product $n \cdot m = (2a + 1)(2b +1) = 4ab + 2a + 2b +1=2( 2ab + a + b) + 1 =
      2k + 1$, where k = (2ab +a +b ) is an integer. Therefore the product of two odd numbers results in another odd
      number. \\
      Since the product of two odds results in an odd number, it follows that $3n$ is odd thus $3n+2$ is odd.
      Therefore, the assumption $3n+2$ is even and $n$ results in a contradiction. In conclusion, if $n$ is an
      integer and $3n+2$ is even then $n$ is even.
\end{enumerate}
\textbf{30.} \\ \ \\
Let $a$ and $b$ be real numbers
\begin{enumerate}[(i)]
\item a is less than b
\item The average of $a$ and $b$ is greater than $a$
\item the average of $a$ and $b$ is less than $b$
\end{enumerate}
$(i) \rightarrow (ii)$


Suppose that $a < b$


$\Rightarrow b>a$


$\Rightarrow b + a > a + a$


$\Rightarrow b + a > 2a$


$\Rightarrow \dfrac{b+a}{2} > \dfrac{2a}{2}$


$\Rightarrow \dfrac{b+a}{2} > a$


$\therefore$ the average of $a$ and $b$ is greater than $a$ \\ \ \\
$(ii) \rightarrow (iii)$


Suppose $\dfrac{a+b}{2} > a$\\


$\Rightarrow a + b > 2a$


$\Rightarrow b > 2a - a$


$\Rightarrow b > a$


$\Rightarrow b + b > a + b$


$\Rightarrow 2b > a + b$


$\Rightarrow b > \dfrac{a + b}{2}$


$\Rightarrow \dfrac{a + b}{2} < b$\\ \ \\
$(iii) \rightarrow (i)$

Suppose $\dfrac{a+b}{2} < b$


$\Rightarrow a + b < 2b$


$\Rightarrow a < 2b - b$


$\Rightarrow a < b$\\ \ \\
\bf Thus the three statements (i), (ii), and (iii) are equivalent.





\end{solution}

\newpage

% Proof Methods and Strategies
\begin{problem} (20 points)\\
Exercises 2, 4, 6, and 8 of Section 1.8 (page 108).
\end{problem}


\begin{solution}\ \\



\textbf{2.} Prove that there are no positive perfect cubes less than 1000 that are the sum of the cubes of two


positive integers. \\

$1^{3} = 1, 2^{3} = 8, 2^{3} = 8, 3^{3} = 27, 4^{3} = 64, 5^{3} = 125, 6^{3} = 216, 7^{3} = 343, 8^{3} = 512, 9^{3} = 729 $

The proof must show that $m^3 + n^3 = a^3$ is false where $a = 1,2,3,4,5,6,7,8,9$ and $m$ and $n$ are positive


integers less than $a$ \\


$m^3 + n^3 = a^3$


$m^3 = a^3 - n^3$


$\therefore m^3 = (a-n)(a^2 +2an + n^2)$ \\


Using $a =2$ and $ n = 1$


$(2-1)(4+2+1) = 7 \neq m^3$ \\

By using a Mathematical computation device, the following is shown


If $a = 3$ then $n = 1,2$ which results in $ \neq m^3$


If $a = 4$ then $n = 1,2,3$ which results in $37, 56, 63 \neq m^3$


If $a = 5$ then $n = 1,2,3,4$ which results in $61,98, 117,124 \neq m^3$


If $a = 6$ then $n = 1,2,3,4$ which results in $91, 152, 189, 208, 215 \neq m^3$


If $a = 7$ then $n = 1,2,3,4,5,6$ which results in $ 127, 218, 279, 316, 335, 342 \neq m^3$


If $a = 8$ then $n = 1,2,3,4,5,6,7$ which results in $ 169, 296, 387, 448, 465, 504, 511\neq m^3$


If $a = 9$ then $n = 1,2,3,4,5,6,7,8$ which results in $ 217, 386, 513, 604, 665, 702, 721, 728\neq m^3$\\


$\therefore$ by proof of exhaustion, it is shown that there are now positive integers whose sum is a perfect cube



less than 1000 \\ \ \\


\textbf{4.} Use a proof by cases to show that min(a, min(b, c)) = min(min(a, b), c) whenever a, b, and c are real


numbers\\


Let $a,b,$and$ c$ be real numbers\\


\textit{Case 1:} Suppose $min(b,c) = b$ and $a \leq b$


Then $min(a,min(b,c))= min(a,b) = a$


Also, $min(a,min(b,c))= min(a,c) = a$ because $min(b,c) = b \wedge a \leq b \rightarrow a \leq c$


$\therefore min(min(a,min(b,c)) = min(min(a,b),c)$\\



\textit{Case 2:} Suppose $min(b,c) = b$ and $b \leq a$


Then $min(a,min(b,c))= min(a,b) = b$


Also, $min(a,min(b,c))= min(a,c) = b$ because $min(b,c) = b \wedge b \leq a \rightarrow b \leq c$


$\therefore min(min(a,min(b,c)) = min(min(a,b),c)$\\


\textit{Case 3:} Suppose $min(b,c) = c$ and $a \leq c$


Then $min(a,min(b,c))= min(a,c) = a$


Also, $min(c,min(b,c))= min(a,c) = a$ because $min(b,c) = a \wedge a \leq c \rightarrow a \leq b$


$\therefore min(min(a,min(b,c)) = min(min(a,b),c)$\\



\textit{Case 4:} Suppose $min(b,c) = c$ and $c \leq a$


Then $min(a,min(b,c))= min(a,c) = c$


Also, $min(c,min(b,c))= min(a,c) = c$ because $min(b,c) = c \wedge c < a \rightarrow a \leq b$


$\therefore min(min(a,min(b,c)) = min(min(a,b),c)$\\


\textbf{6.} Prove using the notion of without loss of generality that 5x + 5y is an odd integer when x and y are


integers of opposite parity\\


\textit{Case 1:} When $x$ is even and $y$ is odd


Then $x = 2m$ and $y = 2n+1$ where $m$ and $n$ are integers

\begin{tabular}{ll}
$5x + 5y$ & $= 5(2m) + 5(2n+1)$ \\
          & $= 10m + 10n + 5$ \\
          & $= 10m + 10n + 4 + 1$ \\
          & $= 2(5m + 5n + 2) + 1$ \\
\end{tabular} \\


$\therefore$ $5x + 5y$ is odd when $x$ is even and $y$ is odd.\\



\textit{Case 2:} When $x$ is odd and $y$ is even


Then $x = 2m +1$ and $y = 2n$ where $m$ and $n$ are integers

\begin{tabular}{ll}
$5x + 5y$ & $= 5(2m+1) + 5(2n)$ \\
          & $= 10m + 10n + 5$ \\
          & $= 10m + 10n + 4 + 1$ \\
          & $= 2(5m + 5n + 2) + 1$ \\
\end{tabular} \\


$\therefore$ $5x + 5y$ is odd when $x$ is odd and $y$ is even.\\


\textbf{8.} Prove that there is a positive integer that equals the sum of the positive integers not exceeding it. Is


your proof constructive or nonconstructive?\\


Suppose that $n=$ sum of positive $n$ integers


$n = 1 + 2 + 3 + ... + n$


$n = \dfrac{n(n+1)}{n}$


$2n = n(n+1)$


$2 = (n+1)$


$n = 1$


$\therefore$ there is a positive integer that equals the sum of the positive integers that will not exceed it


The proof is \bf constructive





\end{solution}

\newpage


\bigskip
\honor

\bigskip
\checklist
\end{document}
